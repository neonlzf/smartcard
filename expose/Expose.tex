\documentclass[parskip]{scrartcl}
\usepackage[utf8]{inputenc}
\usepackage[ngerman]{babel}
\usepackage[round]{natbib}
\usepackage{color} % used for comments
\usepackage{listings}
\usepackage{url}
\lstset {
  language=xml,
  basicstyle={\footnotesize\ttfamily},
  numbers=none,
  aboveskip=5mm,
  belowskip=5mm,
  showstringspaces=false,
  columns=flexible,
  keywordstyle={\bfseries\color{Blue}},
  commentstyle={\color{Red}\textit},
  stringstyle=\color{Magenta},
  frame=single,
  breaklines=true,
  breakatwhitespace=true,
  tabsize=4,
  morekeywords={rdf,rdfs,owl}  % <-- adding custom keywords
}

\begin{document}
	\subject{Projektdokumentation im Fach Smartcard-Programmierung}
	\title{Patientendateninformationskarte im medizinischen Sektor}
	\author{Sebastian Krause, B.Sc.\\Robert Kupferschmied, B.Sc.\\Roy Meissner, B.Sc.}
	\date{\today}
	
	\maketitle
	\vspace{\fill}
	\tableofcontents
	\newpage
	
	\section{Themenbeschreibung}
		Ziel des Projektes ist es, eine Smartcard mit Patientendaten zur Verfügung zu stellen, die im medizinischen Sektor Automatisierung ermöglicht und Arbeitsabläufe unterstützt. Die Smartcard stellt folgende Daten bereit:

		\begin{itemize}
			\item eine Patienten-ID
			\item die Patientenblutgruppe
			\item eine Liste nicht verträglicher Medikamente
			\item eine List regelmäßig einzunehmender Medikamente, incl. Dosierungsinformationen
		\end{itemize}
			
		Die Patientendaten auf der Smartcard können von verschiedenen Rollen(Personen) ausgelesen werden, die jeweils verschiedene Rechte besitzen. Beispielsweise ein Pfleger darf die Liste regelmäßig einzunehmender Medikamente auslesen, der Hausarzt darf hingegen alle Daten auslesen.
		
		Um die Datenübertragung zwischen Lesegerät bzw. Offcard-Software und Smartcard abzusichern, wird RSA benutzt. Das erlaubt sowohl eine Verschlüsselung der Daten, als auch eine Absicherung zwischen den verschiedenen Rollen. 
		
		Als Anwendungsfälle könnten gelten:
	
		\begin{description}
			\item[Notarzt] Im Falle eines Unfalls muss der Notarzt schnell auf Informationen wie Blutgruppe, nicht verträgliche Medikamente und aktuell verschriebene Medikamente zugreifen können, auch wenn der Patient bewusstlos ist. 
			\item[Automatisierte Medikamentenausgabe] Über die Liste regelmäßig einzunehmender Medikamente inklusive Dosierungsinformationen könnten an einem Automaten dem Patienten Medikamente aushändigt werden.
		\end{description}
	
	\section{Programmverteilung}
		Die Applikationen müssen sowohl in eine Oncard- sowie Offcard-Bereich aufgeteilt werden, der jeweils verschiedene Anwendungsfälle umfasst.

		\subsection{Oncard}
		
			\paragraph{Funktionsumfang}
			Die Smartcard enthält folgende Daten, die teils fest einkodiert sind.:
			
			\begin{itemize}
				\item eine Patienten-ID
				\item die Patientenblutgruppe, codiert als Zahl
				\item eine Liste nicht verträglicher Medikamente, bestehend aus Medikamenten-IDs
				\item eine List regelmäßig einzunehmender Medikamente mit Dosierungsinformationen, bestehend aus Medikamente-ID(s), Menge(n) und Zeitpunkt(en)
				\item RSA-Schlüsselpaare sowohl für die Karte selbst, als auch für Nutzerrollen
			\end{itemize}
		
			Zugegriffen werden kann auf die Smartcard über eine Auswahl von Funktionen, die die Daten sowohl manipulieren als auch auslesen können.
			
			\begin{description}
				\item[addToWhitelist] Hinzufügen eines Eintrages zur Liste einzunehmender Medikamente
				\item[removeFromWhitelist] Entfernen eines Eintrages aus der Liste einzunehmender Medikamente
				\item[addToBlacklist] Hinzufügen eines Eintrages zur Liste nicht verträglicher Medikamente
				\item[removeFromBlacklist] Entfernen eines Eintrages von der Liste nicht verträglicher Medikamente
				\item[readWhitelist] Auslesen der vollständigen Liste einzunehmender Medikamente
				\item[readBlacklist] Auslesen der vollständigen Liste nicht verträglicher Medikamente
				\item[readPatientID] Patienten-ID auslesen
				\item[readBloodType] Blutgruppe auslesen
			\end{description}
			
			Bei jedem Aufruf einer Funktion wird die Rollen-ID zur Zugriffssteuerung als Parameter übergeben. Jegliche Kommunikation zwischen Karte und Anwendung wird RSA verschlüsselt.
		
		\subsection{Offcard}
			Die Offcard-Seite ist eine rollenspezifische Anwendung zur Bearbeitung und Visualisierung der auf der Karte verfügbaren Daten. Je nach ausgewählter Rolle können verschiedene Aktionen durchgeführt werden, die dem APDU-Interface der Oncard-Anwendung entsprechen. Je nach Rolle stehen verschieden viele Anwendungsfälle zu Verfügung. Umgesetzt wird der Offcard-Teil durch eine Java-GUI-Anwendung. 
			
			Die Datenhaltung der Medikamente erfolgt, entspechend des beschränkten Umfangs des Beispielprojekts, über eine strukturierte Textdatei. Diese kann beliebig um weitere Medikamente incl. ID erweitert werden.
			
			Die Anwendung bietet die Möglichkeit, sich eine der verfügbaren Rollen auszuwählen und die benötigte Daten von der Smartcard zu lesen sowie anzuzeigen.
		
	\section{Aufgabenverteilung}
	
\end{document}
